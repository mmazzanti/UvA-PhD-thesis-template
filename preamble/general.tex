% File encoding
\usepackage[utf8]{inputenc}

% Languages
\usepackage[main=english]{babel}

% classicthesis setup
\PassOptionsToPackage{%
  %drafting,%
  dottedtoc,
  eulerchapternumbers,
  %listings,%
  %parts,%
  floatperchapter, pdfspacing,%
  beramono,%
  palatino=false,
  %minionprospacing,
  %subfig,%
  %eulermath,%
  b5paper,%
}{classicthesis}

% Meta
%
% Document information
%
\newcommand{\myTitle}{%
  A great PhD \xspace%
}

\newcommand{\mySubTitle}{%
	and its wonderful results\xspace%
}

\newcommand{\myPlainTitle}{%
  Trapped Ions in Optical Tweezers\xspace%
}

\newcommand{\myDissNumber}{update-later}
\newcommand{\myName}{Name Surname\xspace}
\newcommand{\myUni}{\protect{Universiteit van Amsterdam}\xspace}
\newcommand{\myLocation}{Amsterdam\xspace}
\newcommand{\myTime}{Amsterdam, 2024\xspace}

% DOI
\newcommand{\myDOI}{update-later}

% Work version info
\newcommand{\worktodo}[1]{\textcolor{red}{\textbf{TODO}: #1}}
\usepackage{scrtime}
\newcommand{\workdraft}{\textcolor{gray}{Draft, \today{}---\thistime}}
% \renewcommand{\workdraft}{}
% \renewcommand{\worktodo}{}


% Hyphenation
\input{preamble/hyphenation}

% Finetuning
\newcounter{dummy}
\newlength{\abcd}

% Common abbreviations
\newcommand{\ie}{i.\,e.\xspace}
\newcommand{\Ie}{I.\,e.\xspace}
\newcommand{\eg}{e.\,g.\xspace}
\newcommand{\Eg}{E.\,g.\xspace}



% A command for inner product and bras and kets

% Various bracketing commands
\newcommand{\of}[1]{\!\left(#1\right)}
\newcommand{\sqof}[1]{\left[#1\right]}
\newcommand{\cuof}[1]{\left\{#1\right\}}

%Commands for Yb notations
\newcommand{\Yb}[1]{$^{#1}\mathrm{Yb}^+$}
% use "delimited" macros (e.g. with / at the end) as an alternative. The main advantage of \naive/ is that an error message will occur if you happen to forget the closing slash. (https://tex.stackexchange.com/questions/25820/no-space-following-macro-without-argument)
\newcommand{\yb}{}% To make sure that \naive isn't already defined    
\def\yb/{$\mathrm{Yb}^+$}


% Biblatex
\input{preamble/biblatex}

% Redefine cite command to include space before
% <http://tex.stackexchange.com/questions/11602/>
\let\origcite\cite%
\def\cite#1{\unskip~\origcite{#1}}

% Math stuff
\usepackage{amsmath}
%\usepackage{amssymb}
\usepackage{mathtools}
\usepackage{isomath}

% Figures, tables, and captions
\usepackage{tabularx}
\usepackage{ltablex}
\setlength{\extrarowheight}{3pt}
\newcommand{\tableheadline}[1]{\multicolumn{1}{c}{\spacedlowsmallcaps{#1}}}
\newcommand{\myfloatalign}{\centering}
\usepackage{floatrow}
\usepackage{caption}
\captionsetup{format=hang,font=small,labelfont={sc},margin=5pt}
\usepackage{subcaption}
\captionsetup[sub]{margin=0pt,font=small,labelfont={rm}}



% Blind text
%\usepackage{blindtext}
\usepackage{comment}

% Hyperref
\usepackage[dvipsnames]{xcolor}
\usepackage[%
  hyperfootnotes=false,%
  pdfpagelabels,%
  % pdfa,%
  %pdftex,colorlinks=false
]{hyperref}
\usepackage{hyperxmp}
%\pdfcompresslevel=9
%\pdfadjustspacing=1
\hypersetup{%
  %pdfstartpage=3, pdfstartview=FitV,%
  % following line: colored links (web version)
  colorlinks=false, linktocpage=true,%
  % following line: all links in black (for printing)
  %colorlinks=false, linktocpage=false, pdfborder={0 0 0},%
  breaklinks=true, pdfpagemode=UseNone, pageanchor=true, pdfpagemode=UseOutlines,%
  plainpages=false, bookmarksnumbered, bookmarksopen=true, bookmarksopenlevel=1,%
  hypertexnames=true, pdfhighlight=/O,%nesting=true,%frenchlinks,%
  pdftitle={\myPlainTitle},%
  pdfauthor={\myName},%
  pdfcopyright={Copyright (C) \myTime, \myName},%
  pdfsubject={},%
  pdfkeywords={},%
  pdflang={en},%
}

% Graphics
\usepackage{graphicx}
\usepackage{rotating}
\usepackage{tikz}
\usepackage{pgfplots}
\pgfplotsset{compat=newest}
\usetikzlibrary{shapes.geometric}

% Circled numbers
\newcommand*\circled[1]{
  \tikz[baseline=(char.base)]{
    \node[shape=circle,draw,inner sep=1pt,font=\footnotesize,%
          minimum size=0.8\baselineskip] (char) {\figureversion{lining}#1};
  }
}

% This for the chemical formulas used in the gold plating section
\usepackage{chemformula}

% \tikzexternalize
% \tikzsetexternalprefix{externalized/}

% Autoreferences
\renewcommand*{\figureautorefname}{Figure}
\renewcommand*{\tableautorefname}{Table}
\renewcommand*{\partautorefname}{Part}
\renewcommand*{\chapterautorefname}{Chapter}
\renewcommand*{\sectionautorefname}{Section}
\renewcommand*{\subsectionautorefname}{Section}
\renewcommand*{\subsubsectionautorefname}{Section}
\providecommand{\subfigureautorefname}{\figureautorefname}%
\usepackage{cleveref}

% Acronyms
\PassOptionsToPackage{printonlyused}{acronym}
\usepackage{acronym}

% List items
\usepackage{enumitem}

% Load classicthesis style
\usepackage{scrhack}
\usepackage{classicthesis}
%\usepackage{lmodern}
%\usepackage[T1]{fontenc}
%\usepackage[utf8]{inputenc}
% Customize text width and length

\KOMAoptions{headinclude=true,footinclude=false}
%\setlength{\textwidth}{10.5cm} % 9 pt font
\setlength{\textwidth}{13cm} % 10 pt font
% text height set by golden ratio
\areaset[current]{\textwidth}{1.618034\textwidth}

% Page numbers in plain style (chapter titles)
%\clearscrplain
\clearplainofpairofpagestyles
\ofoot[\pagemark]{}
% Adjust distance to footer (default is too large)
\setlength{\footskip}{19pt}

%
% Font setup
%


% Micro-typographic extensions
%\usepackage[protrusion=true,expansion=true]{microtype}

% Use Minion Pro
%\usepackage[
%  mathlf, % lining figures
%]{MinionPro}
%\linespread{1.06}

%
% Customize colors
%
\definecolor{chapter-color}{cmyk}{1, 0.50, 0, 0.25}
\definecolor{link-color}{cmyk}{1, 0.50, 0, 0.25}
\definecolor{cite-color}{cmyk}{0, 0.7, 0.9, 0.2}

% Hyperref
\usepackage{bookmark}
\hypersetup{
  %urlcolor=webbrown, linkcolor=RoyalBlue, citecolor=webgreen, %pagecolor=RoyalBlue,%
  %urlcolor=webbrown, linkcolor=Maroon, citecolor=webgreen,%
  %urlcolor=link-color, linkcolor=link-color, citecolor=cite-color,%
  urlcolor=Black, linkcolor=Black, citecolor=Black, %pagecolor=Black,%
}

% Chapter font
\let\chapterNumber\undefined%
\newfont{\chapterNumber}{eurb10 scaled 5500}
%\newfont{\chapterNumber}{MinionPro-Regular-lf-t1 scaled 5500}

% SI units
\usepackage{siunitx}
\sisetup{
  separate-uncertainty,
  %repeatunits=false,
  detect-family,
  unit-mode=text,
}
\DeclareSIUnit\au{a.u.}
\let\u=\SI%
\newcommand{\angstrom}{\text{\normalfont\AA}}
% Product type codes
\newcommand{\productcode}[1]{\figureversion{lining}#1}

% TOC
\renewcommand{\cftpartfont}{\color{chapter-color}\normalfont}%
\renewcommand{\cftpartpagefont}{\normalfont}%

% Chapter number on inside
\definecolor{chapternumbercolor}{rgb}{0.501961, 0.807843, 0.843137}
\titleformat{\chapter}[display]%
  {\relax}{\vspace*{-3\baselineskip}\makebox[\linewidth][r]{\color{chapternumbercolor}\chapterNumber\thechapter}}{10pt}%
  {\raggedright\spacedallcaps}[\normalsize\vspace*{.8\baselineskip}\titlerule]%

% Chapter abstract
\def\chapterabstract#1{%
  \begingroup
  \baselineskip1.3em
  \leftskip1em
  \rightskip\leftskip\itshape#1
  \par
  \endgroup
}

% Chapter quotes
% Adapted from: <http://tex.stackexchange.com/questions/53377/inspirational-quote-at-start-of-chapter>
\setkomafont{dictumtext}{\itshape\small}%
\setkomafont{dictumauthor}{\normalfont}
\renewcommand*{\dictumwidth}{0.6\textwidth}
\renewcommand*{\dictumrule}{}
\renewcommand*\dictumauthorformat[1]{--- #1}
%\renewcommand{\vec}{\bs}
%\newcommand{\kk}{\vec{k}}


% Subfigure labels (manual)
%\newcommand{\subfig}[1]{#1)}
\newcommand{\subfig}[1]{(#1)}

% CV
\newcommand{\cvleft}[1]{\begin{minipage}[t]{2.5cm}\begin{flushright}#1\end{flushright}\end{minipage}\hspace{5mm}}
\newcommand{\cvright}[1]{\begin{minipage}[t]{8cm}{#1}\end{minipage}}

% Footnote without number
\newcommand\blfootnote[1]{%
  \begingroup
  \renewcommand\thefootnote{}\footnote{#1}%
  \addtocounter{footnote}{-1}%
  \endgroup
}

% Put pages on A4 w/ crop marks
%\usepackage[a4,center,cam]{crop}

% Widow and club penalties
\clubpenalty = 10000
\widowpenalty = 10000
%\displaywidowpenalty = 10000

%stuff coming from the papers
\usepackage{bm}
\usepackage{amssymb}
\usepackage{amsfonts}
\usepackage{listings}
%\usepackage{enumerate}
\usepackage{latexsym}
\usepackage{multirow}
%\usepackage{psfrag}
%\usepackage{bm}
%\usepackage[all]{xy}
\usepackage{braket}
%\usepackage[pdftex,colorlinks=false]{hyperref}
%\usepackage{xcolor}
\usepackage{verbatim}
%\usepackage{float}
%\usepackage{morefloats}%needed to make marginpar behave
%\newcommand{\todo}[1]{$ \color{red} ^{\clubsuit}$ \marginpar{  {\color{red} $ \clubsuit $ \color{black}} \tiny   #1 \color{black}  }}
%\lstloadlanguages{Matlab}
\usepackage{times}
\usepackage{array, makecell}
\usepackage[export]{adjustbox}



% Handy packages
\usepackage{csquotes}  % smart quotes
%\usepackage[T1]{fontenc}
%\usepackage{lmodern}
\usepackage{lmodern}
\usepackage[T1]{fontenc}
\usepackage[utf8]{inputenc}


\usepackage{xspace}
\usepackage{textcomp}
\usepackage{mparhack}
\usepackage{relsize}
%\usepackage[pdftex,colorlinks=false]{hyperref}
%\usepackage{xcolor, colortbl}
\usepackage{colortbl}
\newcommand{\todo}[1]{$ \color{red} ^{\clubsuit}$ \marginpar{  {\color{red} $ \clubsuit $ \color{black}} \tiny   #1 \color{black}  }}





% boldsymbol (requires amsmath)
\newcommand{\bs}[1]{\boldsymbol{#1}}

\newcommand{\red}[1]{{\textcolor{red}{#1}}}
\newcommand{\blue}[1]{{\textcolor{blue}{#1}}}
\newcommand{\magenta}[1]{{\textcolor{magenta}{#1}}}
\newcommand{\green}[1]{{\textcolor[rgb]{0,0.5,0}{#1}}}


% A command for inner product and bras and kets

% Various bracketing commands
%\newcommand{\of}[1]{\!\left(#1\right)}
%\newcommand{\sqof}[1]{\left[#1\right]}
%\newcommand{\cuof}[1]{\left\{#1\right\}}



% commutator and anticommutator
\newcommand{\comm}[2]{\left[#1,#2\right]}
\newcommand{\anticomm}[2]{\left\{#1,#2\right\}}

% 1/2
\newcommand{\half}{$\frac{1}{2}$ }
% simplifies using the up and down arrows to denote spin
\newcommand{\up}{\uparrow}
\newcommand{\dw}{\downarrow}
% Theta function
\newcommand{\tfunc}{\vartheta_1}
% notation for vacuum, an empty set inside a ket
\newcommand{\vac}{\left|\,0\,\right\rangle}
% Absolute value
\newcommand{\abs}[1]{\left|#1\right|}

% Roman functions for real and imaginary parts
\newcommand{\re}{\mathrm{Re}}
\newcommand{\im}{\mathrm{Im}}

% Sets of up-spin and down-spin locations
\newcommand{\bket}{\left\{z_1 \cdots z_{\num}\right\}}
\newcommand{\wket}{\left\{w_1 \cdots w_{\num}\right\}}

%Expectation values
\newcommand{\expect}[1]{\left\langle#1\right\rangle}

% reference with parenthesis
\newcommand{\pref}[1]{(\ref{#1})}


%sign function
\DeclareMathOperator{\sign}{sign}
\usepackage{bbm}

%package for strike-out text (not needed in final version)
\usepackage{ulem}
% this is needed for the usual \emph text (not underlined but italic)
\normalem



% This part to make some authors bold (the thesis author in the list of publications at the beginning)
\def\makenamesetup{%
	\def\bibnamedelima{~}%
	\def\bibnamedelimb{ }%
	\def\bibnamedelimc{ }%
	\def\bibnamedelimd{ }%
	\def\bibnamedelimi{ }%
	\def\bibinitperiod{.}%
	\def\bibinitdelim{~}%
	\def\bibinithyphendelim{.-}}    
\newcommand*{\makename}[2]{\begingroup\makenamesetup\xdef#1{#2}\endgroup}

\newcommand*{\boldname}[3]{%
	\def\lastname{#1}%
	\def\firstname{#2}%
	\def\firstinit{#3}}
\boldname{}{}{}

% Patch new definitions
\renewcommand{\mkbibnamegiven}[1]{%
	\ifboolexpr{ ( test {\ifdefequal{\firstname}{\namepartgiven}} or test {\ifdefequal{\firstinit}{\namepartgiven}} ) and test {\ifdefequal{\lastname}{\namepartfamily}} }
	{\mkbibbold{#1}}{#1}%
}

\renewcommand{\mkbibnamefamily}[1]{%
	\ifboolexpr{ ( test {\ifdefequal{\firstname}{\namepartgiven}} or test {\ifdefequal{\firstinit}{\namepartgiven}} ) and test {\ifdefequal{\lastname}{\namepartfamily}} }
	{\mkbibbold{#1}}{#1}%
}

\boldname{Name}{Surname}{}
\usepackage[left=26mm,right=21mm,top=24.8mm,bottom=26.0mm]{geometry}

% This is needed for dummy text. No purpose in the thesis itself
\usepackage{lipsum} 